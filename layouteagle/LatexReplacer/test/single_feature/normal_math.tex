%! Author = stefan
%! Date = 22.04.20

% Preamble
\documentclass[11pt]{article}

\usepackage{amsthm}
\usepackage{bm}
\usepackage{amsmath}
\usepackage{amssymb}
\usepackage{setspace}
\usepackage{hyperref}
%\usepackage{esint}
\PassOptionsToPackage{normalem}{ulem}
\usepackage{ulem}
%\onehalfspacing

\newcommand{\noprint}[1]{}
\newcommand{\destacar}[1]{{\bfseries\slshape\color{red} #1}}
\newcommand{\destacarj}[1]{{\bfseries\slshape\color{blue} #1}}
\newcommand{\destacare}[1]{{\bfseries\slshape\color{green} #1}}
\newcommand{\tr}{{\text{tr}}}
\newcommand{\ft}{\tilde}
\newcommand{\ii}{\mathrm{i}}
\renewcommand{\d}{\mathrm{d}}
\renewcommand{\Re}{\mathrm{Re}}
\renewcommand{\Im}{\mathrm{Im}}
\newcommand{\nn}{\nonumber}
\newcommand{\edu}[1]{{\color{magenta} #1}}
\newcommand{\educ}[1]{\textbf{{\color{magenta} [EDU: #1]}}}
\newcommand{\pipo}[1]{{\color{red} #1}}
\newcommand{\pipoc}[1]{\textbf{{\color{red} [Pipito: #1]}}}

\newtheorem{defi}{Definition}
\newtheorem{teor}{Theorem}
%%%%%%%%%%%%%%%%%%%%%%%%%%%%%% User specified LaTeX commands.
\usepackage{braket}
\usepackage{bbm}

\providecommand{\abs}[1]{\lvert#1\rvert}
\providecommand{\norm}[1]{\lVert#1\rVert}
%para comandos de valor absoluto y norma


% Packages
\usepackage{amsmath}

% Document
\begin{document}

We consider a two-level quantum system subject to repeated interactions with an environment. Namely, the interaction of the system and the environment will be switched on and off for a finite number of times.  Consider that the evolution of  the system and the environment is described by the rather general Hamiltonian
\begin{align}
\hat{H}&=\hat{H}_{\textsc{s}}+\hat{H}_{\textsc{E}}+\hat{H}_{\textsc{se}},
\end{align}
where  $\hat{H}_{\textsc{E}}$ denotes the free evolution of the environment, which we leave arbitrary.  $\hat{H}_{\textsc{s}}=\Omega \mathbf{h}\cdot \bm{\hat{\sigma}}$ is the free Hamiltonian of the two-level system. The spin-environment interaction Hamiltonian is given by
\begin{equation}
    \hat{H}_{\textsc{se}}= \chi(t) \bm{\alpha}\cdot \bm{\hat{\sigma}} \otimes \hat{{O}}.
\end{equation}
Here  $\mathbf{h}\cdot \bm{\hat{\sigma}}$ and $\bm{\alpha}\cdot \bm{\hat{\sigma}}$ are Pauli observables in the directions $\mathbf{h}$ and $\bm{\alpha}$ respectively.  $\hat{{O}}$ is an observable of the environment, that is, a self-adjoint operator that also commutes with all observables of the spin.

Finally $\chi(t)$ is a switching function that regulates the rate and intensity of the interactions between the two-level system and the environment. Note that outside the support of $\chi$ the two subsystems evolve independently.
\end{document}