%! Author = stefan
%! Date = 27.04.20

% Preamble
\documentclass[letterpaper, 10 pt, conference]{ieeeconf}
% Packages
\usepackage{amsmath}




\title{\LARGE \bf
Covariance intersection to improve the robustness of the photoplethysmogram derived respiratory rate}
%\title{\LARGE \bf
%Improving accuracy and retention rate of respiratory rate estimation from Photoplethmogram}


\author{Jia Zhang, Gaetano Scebba and Walter Karlen, \it{Senior Member}, IEEE% <-this % stops a space
\thanks{The work was supported by the Swiss National Science Foundation SNSF (150640).}% <-this % stops a space
\thanks{Mobile Health Systems Lab, Institute of Robotics and Intelligent Systems, Department of Health Sciences and Technology, ETH Zurich, Zurich, Switzerland (email:
		{\tt\small jia.zhang@hest.ethz.ch},
        {\tt\small walter.karlen@ieee.org})}%
}


% Document
\begin{document}

    \maketitle

\thispagestyle{empty}
\pagestyle{empty}


%%%%%%%%%%%%%%%%%%%%%%%%%%%%%%%%%%%%%%%%%%%%%%%%%%%%%%%%%%%%%%%%%%%%
\begin{abstract}
Respiratory rate (RR) can be estimated from the photoplethysmogram (PPG) recorded by optical sensors in wearable devices. The fusion of estimates from different PPG features has lead to an increase in accuracy, but also reduced the numbers of available final estimates due to discarding of unreliable data. We propose a novel, tunable fusion algorithm using covariance intersection to estimate the RR from PPG (CIF). The algorithm is adaptive to the number of available feature estimates and takes each estimates' trustworthiness into account. In a benchmarking experiment using the CapnoBase dataset with reference RR from capnography, we compared the CIF  against the state-of-the-art Smart Fusion (SF) algorithm. The median root mean square error was 1.4 breaths/min for the CIF and 1.8 breaths/min for the SF. The CIF significantly increased the retention rate distribution of all recordings from 0.46 to 0.90 (p~$<$~0.001). The agreement with the reference RR was high with a Pearson's correlation coefficient of 0.94, a bias of 0.3 breaths/min, and limits of agreement of -4.6 and 5.2 breaths/min. In addition, the algorithm was computationally efficient. Therefore, CIF could contribute to a more robust RR estimation from  wearable PPG recordings.

\end{abstract}


%%%%%%%%%%%%%%%%%%%%%%%%%%%%%%%%%%%%%%%%%%%%%%%%%%%%%%%%%%%%%%%%%%%%%%%%%%%%%%%%
\section{INTRODUCTION}

%1. abnormal detection, continuous monitoring
%\newline
%2. wearable devices provide the possibility to continuous monitor physiological signals, such as PPG RR can be extracted from PPG, method, state of the art
%\newline
%4. shortness
%\newline
%5. goals of the study
%\newline

Respiratory rate (RR) is an essential vital sign to assess the medical condition of patients and abnormal RR is an important predictor of serious illness \cite{Jayaraman2008}. Continuous monitoring of RR and RR trend changes can detect abnormal events among general ward patients and enable early interventions \cite{Mok2015}. %over time
Commonly used sensing methods in the clinic, such as capnometry and spirometry, are based on the analysis of ex- and inhaled air (i.e. gas flow and composition changes), but are cumbersome to wear, obstructive, or subject to strong artefacts if the environment is not controlled. Wearable devices show great potential for unobstructive and continuous monitoring and could overcome some of these limitations.

Objectively assessed RR with mobile sensors shows potential for improving the diagnosis of acute lower respiratory infections at the point-of-care \cite{Karlen2014}. RR can be estimated by analyzing the photoplethysmogram (PPG), which is increasingly available on wearable devices. The PPG waveform is known to have multiple modulations induced by respiration, such as respiratory-induced intensity (RIIV), amplitude (RIAV), frequency (RIFV) \cite{Karlen2013c}, width (RIWV)\cite{Lazaro2013b}, and slope transit time variation (RISV) \cite{Addison2016c}. The current most cited benchmark method for RR estimation from the PPG signal is Smart Fusion (SF) \cite{Karlen2013c}. It fuses three respiratory-induced variations (RIIV, RIAV, RIFV) by calculating their mean and discards RR estimates that are labelled as artefact or where the standard deviation between the three estimates is larger than $4$ breaths/min.



\end{document}